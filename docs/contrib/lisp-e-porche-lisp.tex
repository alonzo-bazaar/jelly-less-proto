% Created 2023-10-07 Sat 11:53
% Intended LaTeX compiler: pdflatex
\documentclass[11pt]{article}
\usepackage[utf8]{inputenc}
\usepackage[T1]{fontenc}
\usepackage{graphicx}
\usepackage{longtable}
\usepackage{wrapfig}
\usepackage{rotating}
\usepackage[normalem]{ulem}
\usepackage{amsmath}
\usepackage{amssymb}
\usepackage{capt-of}
\usepackage{hyperref}
\usepackage{color}
\usepackage{listings}
\author{Biggie Dickus}
\date{\today}
\title{}
\hypersetup{
 pdfauthor={Biggie Dickus},
 pdftitle={},
 pdfkeywords={},
 pdfsubject={},
 pdfcreator={Emacs 30.0.50 (Org mode 9.6.7)}, 
 pdflang={English}}

% Setup for code blocks [1/2]

\usepackage{fvextra}

\fvset{%
  commandchars=\\\{\},
  highlightcolor=white!95!black!80!blue,
  breaklines=true,
  breaksymbol=\color{white!60!black}\tiny\ensuremath{\hookrightarrow}}

% Make line numbers smaller and grey.
\renewcommand\theFancyVerbLine{\footnotesize\color{black!40!white}\arabic{FancyVerbLine}}

\usepackage{xcolor}

% In case engrave-faces-latex-gen-preamble has not been run.
\providecolor{EfD}{HTML}{f7f7f7}
\providecolor{EFD}{HTML}{28292e}

% Define a Code environment to prettily wrap the fontified code.
\usepackage[breakable,xparse]{tcolorbox}
\DeclareTColorBox[]{Code}{o}%
{colback=EfD!98!EFD, colframe=EfD!95!EFD,
  fontupper=\footnotesize\setlength{\fboxsep}{0pt},
  colupper=EFD,
  IfNoValueTF={#1}%
  {boxsep=2pt, arc=2.5pt, outer arc=2.5pt,
    boxrule=0.5pt, left=2pt}%
  {boxsep=2.5pt, arc=0pt, outer arc=0pt,
    boxrule=0pt, leftrule=1.5pt, left=0.5pt},
  right=2pt, top=1pt, bottom=0.5pt,
  breakable}

% Support listings with captions
\usepackage{float}
\floatstyle{plain}
\newfloat{listing}{htbp}{lst}
\newcommand{\listingsname}{Listing}
\floatname{listing}{\listingsname}
\newcommand{\listoflistingsname}{List of Listings}
\providecommand{\listoflistings}{\listof{listing}{\listoflistingsname}}


% Setup for code blocks [2/2]: syntax highlighting colors

\newcommand\efstrut{\vrule height 2.1ex depth 0.8ex width 0pt}
\definecolor{EFD}{HTML}{000000}
\definecolor{EfD}{HTML}{ffffff}
\newcommand{\EFD}[1]{\textcolor{EFD}{#1}} % default
\definecolor{EFh}{HTML}{7f7f7f}
\newcommand{\EFh}[1]{\textcolor{EFh}{#1}} % shadow
\definecolor{EFsc}{HTML}{228b22}
\newcommand{\EFsc}[1]{\textcolor{EFsc}{\textbf{#1}}} % success
\definecolor{EFw}{HTML}{ff8e00}
\newcommand{\EFw}[1]{\textcolor{EFw}{\textbf{#1}}} % warning
\definecolor{EFe}{HTML}{ff0000}
\newcommand{\EFe}[1]{\textcolor{EFe}{\textbf{#1}}} % error
\definecolor{EFc}{HTML}{b22222}
\newcommand{\EFc}[1]{\textcolor{EFc}{#1}} % font-lock-comment-face
\definecolor{EFcd}{HTML}{b22222}
\newcommand{\EFcd}[1]{\textcolor{EFcd}{#1}} % font-lock-comment-delimiter-face
\definecolor{EFs}{HTML}{8b2252}
\newcommand{\EFs}[1]{\textcolor{EFs}{#1}} % font-lock-string-face
\definecolor{EFd}{HTML}{8b2252}
\newcommand{\EFd}[1]{\textcolor{EFd}{#1}} % font-lock-doc-face
\definecolor{EFm}{HTML}{008b8b}
\newcommand{\EFm}[1]{\textcolor{EFm}{#1}} % font-lock-doc-markup-face
\definecolor{EFk}{HTML}{9370db}
\newcommand{\EFk}[1]{\textcolor{EFk}{#1}} % font-lock-keyword-face
\definecolor{EFb}{HTML}{483d8b}
\newcommand{\EFb}[1]{\textcolor{EFb}{#1}} % font-lock-builtin-face
\definecolor{EFf}{HTML}{0000ff}
\newcommand{\EFf}[1]{\textcolor{EFf}{#1}} % font-lock-function-name-face
\definecolor{EFv}{HTML}{a0522d}
\newcommand{\EFv}[1]{\textcolor{EFv}{#1}} % font-lock-variable-name-face
\definecolor{EFt}{HTML}{228b22}
\newcommand{\EFt}[1]{\textcolor{EFt}{#1}} % font-lock-type-face
\definecolor{EFo}{HTML}{008b8b}
\newcommand{\EFo}[1]{\textcolor{EFo}{#1}} % font-lock-constant-face
\definecolor{EFwr}{HTML}{ff0000}
\newcommand{\EFwr}[1]{\textcolor{EFwr}{\textbf{#1}}} % font-lock-warning-face
\newcommand{\EFnc}[1]{#1} % font-lock-negation-char-face
\definecolor{EFpp}{HTML}{483d8b}
\newcommand{\EFpp}[1]{\textcolor{EFpp}{#1}} % font-lock-preprocessor-face
\newcommand{\EFrc}[1]{\textbf{#1}} % font-lock-regexp-grouping-construct
\newcommand{\EFrb}[1]{\textbf{#1}} % font-lock-regexp-grouping-backslash
\newcommand{\EFob}[1]{#1} % org-block
\definecolor{EFhn}{HTML}{008b8b}
\newcommand{\EFhn}[1]{\textcolor{EFhn}{#1}} % highlight-numbers-number
\definecolor{EFhq}{HTML}{9370db}
\newcommand{\EFhq}[1]{\textcolor{EFhq}{#1}} % highlight-quoted-quote
\definecolor{EFhs}{HTML}{008b8b}
\newcommand{\EFhs}[1]{\textcolor{EFhs}{#1}} % highlight-quoted-symbol
\definecolor{EFrda}{HTML}{707183}
\newcommand{\EFrda}[1]{\textcolor{EFrda}{#1}} % rainbow-delimiters-depth-1-face
\definecolor{EFrdb}{HTML}{7388d6}
\newcommand{\EFrdb}[1]{\textcolor{EFrdb}{#1}} % rainbow-delimiters-depth-2-face
\definecolor{EFrdc}{HTML}{909183}
\newcommand{\EFrdc}[1]{\textcolor{EFrdc}{#1}} % rainbow-delimiters-depth-3-face
\definecolor{EFrdd}{HTML}{709870}
\newcommand{\EFrdd}[1]{\textcolor{EFrdd}{#1}} % rainbow-delimiters-depth-4-face
\definecolor{EFrde}{HTML}{907373}
\newcommand{\EFrde}[1]{\textcolor{EFrde}{#1}} % rainbow-delimiters-depth-5-face
\definecolor{EFrdf}{HTML}{6276ba}
\newcommand{\EFrdf}[1]{\textcolor{EFrdf}{#1}} % rainbow-delimiters-depth-6-face
\definecolor{EFrdg}{HTML}{858580}
\newcommand{\EFrdg}[1]{\textcolor{EFrdg}{#1}} % rainbow-delimiters-depth-7-face
\definecolor{EFrdh}{HTML}{80a880}
\newcommand{\EFrdh}[1]{\textcolor{EFrdh}{#1}} % rainbow-delimiters-depth-8-face
\definecolor{EFrdi}{HTML}{887070}
\newcommand{\EFrdi}[1]{\textcolor{EFrdi}{#1}} % rainbow-delimiters-depth-9-face
\begin{document}

\tableofcontents

\section{Le basi}
\label{sec:orgbe6922f}
Lisp è un linguaggio di programmazione caratterizzato, tra i molti altri, da due fatti
\begin{itemize}
\item è stato fatto negli anni 60
\item è stato fatto (tra gli altri) da matematici\footnote{che l'informatica non esisteva troppo all'epoca come disciplina}
\end{itemize}

è abbastanza uno scempio per chi non ha passato anni ed anni a sviluppare varii strati di sindrome di stoccolma verso di esso, quindi è comprensibile che possa farvi schifo

per vostra sfortuna lisp non è un singolo linguaggio, ma è una famiglia di linguaggi, tutti basati sulle stesse 2/3 idee di base, ma ci hanno fatti tutti il cazzo che volevano
giusto per dire, ecco da \href{https://en.wikipedia.org/wiki/Lisp\_(programming\_language)}{wikipedia} la timeline dei lisp

\begin{center}
\includegraphics[width=.9\linewidth]{/home/big/Pictures/Screenshots/clippato.png}
\end{center}

Lo schifo che intendo fare sarà più uno scheme fatto male con un paio di cose fottute da common lisp.
Scheme perchè scheme è quello più facile da approssimare (lo standard del linguaggio sono sulle 80 pagine, e la gente (inclusi gli autori dello standard) si fa seghe su quanto sia corto lo standard)

\subsection{Ma sto lisp com'è fatto?}
\label{sec:org5e935bb}
in lisp, quasi tutto è una lista, una lista è fatta così
\begin{Code}
\begin{Verbatim}
\color{EFD}\EFrda{(} roba roba roba \EFrda{)}
\end{Verbatim}
\end{Code}

con quasi tutto si intende che anche il codice è una lista, ad esempio, un \texttt{if} è una lista messa così
\begin{Code}
\begin{Verbatim}
\color{EFD}\EFrda{(}\EFk{if} <condizione> <qua il then> <qua l'else>\EFrda{)}
\end{Verbatim}
\end{Code}
ai fini di renderci la vita ancora peggiore, \texttt{<condizione>} non viene testato se è vero o falso, ma se è \texttt{nil}\footnote{nil sarebbe sia il \texttt{false} che il \texttt{null} del lisp} o meno (noto anche come "non nil")

e una chiamata a funzione è una lista messa così
\begin{Code}
\begin{Verbatim}
\color{EFD}\EFrda{(}<funzione> <argomento>*\EFrda{)}
\end{Verbatim}
\end{Code}
con funzione si intende anche roba stra builtin come \texttt{+}, \texttt{=}, \texttt{/}, \ldots{}, ad esempio per sommare due(o più) numeri si fa
\begin{Code}
\begin{Verbatim}
\color{EFD}\EFrda{(}+ 2 3\EFrda{)}
\EFrda{(}+ 2 3 4 5\EFrda{)}
...
\end{Verbatim}
\end{Code}

se vogliamo controllare se \texttt{x == y} si fa\footnote{all'epoca \texttt{==} non era ancora diventato il simbolo del "sono uguali?", i primi standard di questo affare, essendo anni 60, sono antecedenti al C, che era più anni 70}
\begin{Code}
\begin{Verbatim}
\color{EFD}\EFrda{(}= x y\EFrda{)}
\end{Verbatim}
\end{Code}

o per dire \texttt{square(x)}, si fa
\begin{Code}
\begin{Verbatim}
\color{EFD}\EFrda{(}square x\EFrda{)}
\end{Verbatim}
\end{Code}

mettiamo di voler dire "se x è dispari \(print(x^{2})\), altrimeti \(print(\frac{x}{2})\)"
questo si tradurrebbe con
\begin{Code}
\begin{Verbatim}
\color{EFD}\EFrda{(}\EFk{if} \EFrdb{(}= 0 \EFrdc{(}mod x 2\EFrdc{)}\EFrdb{)}
    \EFrdb{(}print \EFrdc{(}/ x 2\EFrdc{)}\EFrdb{)}
  \EFrdb{(}print \EFrdc{(}* x x\EFrdc{)}\EFrdb{)}\EFrda{)}
\end{Verbatim}
\end{Code}

\subsection{Qualche cagata da tenere a mente}
\label{sec:orgd6ca674}
TBD

\subsection{Perchè lisp?}
\label{sec:orgb4c5200}
Due motivi
\begin{itemize}
\item mi piace il linguaggio\footnote{comunque versioni un po' più moderne di quella fatta qui, con cose bellissime quali struct, classi, e ambienti di sviluppo decenti}
\item è una cagata farlo\footnote{rispetto ad altri interpreter, poi grazialcazzo che fare un interpreter è comunque un dito in culo}
\end{itemize}
\subsubsection{Parsing}
\label{sec:orgdbc8b27}
Fare parsing di lisp è una minchiata, la sintassi è all'incirca (si avvisa il lettore che questo non è mai nella vita un ebnf valido, dovrebbe solo rendere l'idea)
\begin{verbatim}
lettera :: [a-z] | [A-Z]
cifra :: [0-9]
simbolo :: (<qualsiasi carattere che non sia uno spazio>)*
stringa :: " <qualsiasi carttere>* "
numero :: <cifra>*
espressione :: simbolo | stringa | numero | ( <espressione>* )
\end{verbatim}

e \href{https://www.softwarepreservation.org/projects/LISP/book/LISP\%201.5\%20Programmers\%20Manual.pdf}{certi manuali}\footnote{vecchi quanto la merda ma comunque} iniziano, giusto per, definendo l'intera sintassi ed evaluator del linguaggio, perchè gli andava\footnote{e perchè il manuale l'aveva scritto il primo autore del linguaggio\footnotemark}\footnotetext[8]{\label{org6240c76}più dello standard/definizione, il codice l'hanno fatto vari suoi studenti che programmavano decisamente meglio di lui, lui insegnava elettronica credo}

\subsubsection{Evaluator}
\label{sec:org53e6f0d}
le regole di valutazione del lisp sono anch'esse abbastanza una cagata da descrivere, ci vorrano sui 2/3 check neanche per capire se un'espressione è un if, un loop, o un che, e da lì mezzo secondo e l'hai già scomposta, ez.
\end{document}